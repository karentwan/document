\documentclass[UTF8]{rjthesis}


\rjhead{陈翔兄 等:静态软件缺陷预测方法研究}

\rjtitle{静态软件缺陷预测方法\textsuperscript{*}}
\rjauthor{陈  翔\textsuperscript{1,2},  顾  庆\textsuperscript{2},  刘望舒,  刘树龙,  倪  超}
\rjinfor{\textsubscript{1}(南通大学 计算机科学与技术学院,江苏 南通  226019)\\
	\textsubscript{2}(计算机软件新技术国家重点实验室(南京大学),江苏 南京  210023)\\
	通讯作者: 陈翔, E-mail: ****@ntu.edu.cn
}

\begin{document}
	
	\rjmaketitle
	\begin{rjabstract}
		静态软件缺陷预测是软件工程数据挖掘领域中的一个研究热点.通过分析软件代码或开发过程,设计出与软件缺陷相关的度量元;随后,通过挖掘软件历史仓库来创建缺陷预测数据集,旨在构建出缺陷预测模型,以预测出被测项目内的潜在缺陷程序模块,最终达到优化测试资源分配和提高软件产品质量的目的.对近些年来国内外学者在该研究领域取得的成果进行了系统总结.首先,给出了研究框架并识别出了影响缺陷预测性能的3个重要影响因素:度量元的设定、缺陷预测模型的构建方法和缺陷预测数据集的相关问题;接着,依次总结了这3个影响因素的已有研究成果;随后,总结了一类特殊的软件缺陷预测问题(即,基于代码修改的缺陷预测)的已有研究工作;最后,对未来研究可能面临的挑战进行了展望.
	\end{rjabstract}
	\rjkeywords{软件质量保障;软件缺陷预测;软件度量元;机器学习;数据集预处理}
	
	% 开始正文
	软件缺陷(software defect)产生于开发人员的编码过程,需求理解不正确、软件开发过程不合理或开发人员的经验不足,均有可能产生软件缺陷.而含有缺陷的软件在运行时可能会产生意料之外的结果或行为,严重的时候会给企业造成巨大的经济损失,甚至会威胁到人们的生命安全.在软件项目的开发生命周期中,检测出内在缺陷的时间越晚,修复该缺陷的代价也越高.尤其在软件发布后,检测和修复缺陷的代价将大幅度增加.因此,项目主管借助软件测试或代码审查等软件质量保障手段,希望能够在软件部署前尽可能多地检测出内在缺陷.但是若关注所有的程序模块会消耗大量的人力物力,因此,项目主管希望能够预先识别出可能含有缺陷的程序模块,并对其分配足够的测试资源.

	本文第1节对静态软件缺陷预测方法的研究框架进行总结,并识别出其中3个重要的影响因素(即,度量元的设定、缺陷预测模型的构建方法和缺陷预测数据集的相关问题).第2节对已有的度量元设计进行总结.第3节对已有的缺陷预测模型构建方法进行总结.第4节对缺陷预测数据集相关问题的产生根源和解决方法进行总结.第5节对一类特殊的软件缺陷预测问题(即,基于代码修改的缺陷预测)的已有研究工作进行总结.传统的缺陷预测问题重点预测的是程序模块内部是否含有缺陷,而该问题的特殊之处在于需要预测出提交的代码修改是否会产生缺陷.最后总结全文,并对未来值得关注的研究方向进行初步探讨.
	
	\section{研究框架}
	静态软件缺陷预测可以将程序模块的缺陷倾向性、缺陷密度或缺陷数设置为预测目标.以预测模块的缺陷倾向性为例,其典型研究框架如图1所示
	\subsection{挖掘历史仓库}
	\section{第二部分}
	图1上半部分总结的是软件缺陷预测过程,该过程主要包括两个阶段:模型构建阶段和模型应用阶段.具体来说,模型构建阶段包括3个步骤.
实在是搞不清楚这是啥意思
	\begin{enumerate}
		\item 第一件事
		\item 第二件事
	\end{enumerate}

\end{document}